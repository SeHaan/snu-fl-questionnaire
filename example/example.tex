\documentclass{snu-fl-questionnaire}

% Hangul font setup
\setmainhangulfont{KoPubWorldBatang_Pro}[
  Scale = MatchUppercase,
  UprightFont={* Light},
  BoldFont={* Bold},
  AutoFakeSlant = 0.15
]
\setsanshangulfont{KoPubWorldDotum_Pro}[
  Scale = MatchUppercase,
  BoldFont={* Bold},
]

% Metadata
\title{2025 한국 언어조사 질문지}
\author{서울대학교 인문대학 언어학과}
\date{2025년 10월 31일 \textasciitilde{} 11월 1일}
\printdate{2025년 10월 31일}
\issuedate{2025년 10월 31일}
\tel{(02) 880-6163, 6164}


\begin{document}

\frontmatter
\maketitle
\tableofcontents

\chapter{2025학년도 언어조사 개요}
2025학년도 언어조사의 일정, 장소, 교통편 등을 적습니다.

\chapter{2025학년도 언어조사 일정}
2025학년도 언어조사의 구체적인 일정을 적습니다.

\chapter{자료제공인 조사표 (1)}
자료제공인 \#1의 정보를 적습니다.

\chapter{자료제공인 조사표 (2)}
자료제공인 \#2의 정보를 적습니다.

\chapter{자료제공인 조사표 (3)}
자료제공인 \#3의 정보를 적습니다.

\chapter{자료제공인 조사표 (4)}
자료제공인 \#4의 정보를 적습니다.

\chapter{일러두기}
일러둘 내용을 적습니다.


\mainmatter
\chapter{어휘·음운편}
\subsection{일러두기}
일러둘 내용을 적습니다.

\section{절 예시}
\subsection{일러두기}
일러둘 내용을 적습니다.

\subsection{자유발화 질문}
자유발화 질문을 적습니다.

\subsection{}
목표 어휘와 예상 형태, 질문을 적습니다.


\chapter{문법편}
\subsection{일러두기}
일러둘 내용을 적습니다.

\section{절 예시}
\subsection{일러두기}
일러둘 내용을 적습니다.

\subsection{소절 예시}
목표 문법 표지와 질문을 적습니다.


\chapter{어휘 의미지도}
\section{절 예시}
주제별로 의미지도를 그립니다.


\chapter{사진 자료}
참고할 사진을 넣습니다.


\chapter*{부록}
\begin{appendices}

\section{어휘 체크리스트}
어휘 체크리스트를 적습니다.


\section{문법 체크리스트}
문법 체크리스트를 적습니다.

\end{appendices}


\backmatter
\makebackcover

\end{document}